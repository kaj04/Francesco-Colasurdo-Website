\title{Actual Goals}
\author{Francesco Colasurdo}
\date{\today}

\begin{document}

\maketitle

\begin{abstract}
Questa pagina descrive gli obiettivi attuali, le priorità e i progressi che sto perseguendo,
presentati con una formattazione in stile accademico per mantenere ordine e rigore espositivo.
\end{abstract}

\section{Obiettivi a breve termine}
\begin{itemize}
  \item Completare il setup del sito web personale basato su Hugo e GitHub Pages.
  \item Pubblicare un articolo a settimana nella sezione \textit{Blog}.
  \item Integrare un template \LaTeX{} personalizzato per le pubblicazioni future.
\end{itemize}

\section{Focus accademico}
Sto approfondendo argomenti di \textbf{apprendimento automatico} e \textbf{ottimizzazione}.
Un obiettivo pratico è implementare e comprendere meglio il gradiente:
\[
\nabla_\theta J(\theta) = \mathbb{E}_{s,a\sim\pi_\theta} 
\big[\nabla_\theta \log \pi_\theta(a|s) Q^{\pi_\theta}(s,a)\big].
\]

\section{KPI e traguardi}
\begin{tabular}{ll}
\textbf{Settimana} & \textbf{Obiettivo principale} \\
\hline
41 & Pubblicare il primo post completo \\
42 & Revisionare il CV in formato \LaTeX{} \\
43 & Definire il primo mini-progetto di RL \\
\end{tabular}

\section{Note finali}
Questo documento è generato da sorgente \LaTeX{} e convertito automaticamente in HTML
tramite \textit{GitHub Actions + Pandoc}, mantenendo la struttura e la matematica originale.

\end{document}
